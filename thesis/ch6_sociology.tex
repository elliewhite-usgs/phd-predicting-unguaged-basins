\chapter[Beyond McDonaldization]{Beyond McDonaldization: The ``Robotanization" of Agriculture} \label{ch6:sociology}
\chaptermark{Beyond McDonaldization}
\setlength{\epigraphwidth}{4.5in}
\epigraph{No increase in material wealth will compensate for arrangements which insult people's self-respect and impair their freedom.}{Richard H. Tawney, \textit{``Religion and the Rise of Capitalism''}, 1926}

%------------------------------------------------------------------------------------------------------------------------------------------------------------------------------------------------------
\section*{Summary} 
Given the increasingly popular statistical learning methods and their resampling techniques, the sociological effects of these methods become important fro for both scholarship and application. In 2013, Monsanto paid \$930 Million USD to acquire The Climate Corporation \cite{frobes2013merger}. Monsanto is a large publicly traded agricultural multinational corporation, and The Climate Corporation is a digital agriculture company that examines field data. In other words, ``big-farma" has learned the potential of big data. Recent technological advances in big data is impacting the nature of agricultural (i.e., arable, livestock, horticulture, and fishery farming) work in various and complex ways. 

This chapter will explore the impacts of big data (BD), machine learning (ML), and artificial intelligence (AI) on arable farming and studies of farming. BD refers to data sets that are too large or complex for traditional data-processing application software to adequately deal with \cite{wiki2018bigdata}. ML is the development of computer programs that analyzes data and automatically learns and improves from experience without being explicitly programmed \cite{koza1996automated}.  ML is an application in the larger field of AI. BD, ML, and/or AI in farming, hereinafter referred to as ``robotany", manifest as a combination of data-heavy decision making and precision agriculture. 

Part of my research will be preparing a literature review on historical and recent mechanization trends in agriculture (e.g., the tomato harvester saving the California tomato) and the sociological process of McDonaldization. My research will cover these topics in the Science and Technology Studies discipline and the Cybernetics sub-discipline.

%-----------------------------------------------------------------------------------------------------------------------------------------------------------------------------------------------------
\section{Introduction} 
The major influences in the development of the first tomato picking harvester (See Figure \ref{fig:tomatoharvester}) were technologies such as, effective machines, specially bred tomatoes, careful irrigation and fertilization, and particular planting techniques, another major influence was a societal phenomenon: the fear of a lack of labor to handle the tomato crop, as the Brarcero migrant-worker program had ended \cite{rasmussen1968advances}. 

The two major hurdles to developing a mechanical tomato picker were the susceptibility of the tomato to bruising, thus making it a tough candidate for machine harvesting, and the fact that tomatoes did not all ripen at the same time, that is, they usually required multiple passes through the field. In the 1940s, UC Davis researchers successfully removed both obstacles; they developed a pear-shaped tomato particularly adapted to machine harvesting that would also ripen around the same time \cite{rasmussen1968advances}. Uniformity in shape and ripening time, brings the predictability needed for mechanization. Predictability is one aspect of McDonaldization. 

% For a brief history of farming (where we were) see Appendix \ref{g:history}. In the following sections, recent improvements in agricultural technology and farming in this era are described (where we are now). 

\begin{figure}[ht]
	\centering
	\includegraphics[width=\textwidth,trim={0 0 0 0},clip=true]{Plots/tomato-harvester-blackwelder-uc.jpg}
	\caption{Blawelder tomato harvester makes fundamental changes to the way scientists think about plants; the science was then less about the delicious tomato but about the tomato that can pass through the machine and get to the market \protect\citeA{fell2016discoveries}} 
	\label{fig:tomatoharvester}
\end{figure}

The more recent tools continuing the tradition of mechanization (i.e., robotany) can be categorized based on their designed functions: (a) crop management, including applications on yield prediction, disease detection, weed detection, crop quality, and species recognition; (b) livestock management, including applications on animal welfare and livestock production; (c) water management; and (d) soil management \cite{liakos2018machine}. The author claims that tools developed in these areas are helping farmers enhance yields, improve efficiency, and manage their risk. Efficiency is another dimension of McDonaldization. 

McDonaldization has four dimensions: efficiency, calculability, predictability, and control \cite{ritzer2002introduction}. \citeA{ritzer2009mcdonaldization} uses the meat industry to describe the four dimensions of this process. In this paper, the organic movement is introduced as an alternative to the industrial model and a rebellion against McDonaldization. After discussing the irrationalities of McDonaldization, the author acknowledged the positive outcomes of this process: abundance of cheap food and the availability of products year round. The author did not speculate on whether McDonaldization will win over the organic food movement or if organic food will ever become mainstream but ends with acknowledging the risks (i.e., externalities or irrationalities) that are being ignored in the process and asks: could it be that the long arm of McDonaldization is reaching too far? \cite{ritzer2009mcdonaldization}

%-----------------------------------------------------------------------------------------------------------------------------------------------------------------------------------------------------
\section{Two Case Studies} 
Large agribusinesses deploy and operationalize most technologies in robotany. These technologies are usually proprietary and patented. For example, IntelinAir, with one headquarter in San Jose, California, has developed drones and airplanes with MRI-like imaging, called Ag-MRI, to help identify anomalies within a field. Therefore, interventions (e.g., the application of chemicals and water) can be applied discerningly. 

There are much fewer examples of small farmers using robotany. In one example, Makoto Koike, a former designer in the Japanese automobile industry and the son of cucumber farmers, used Google's open source ML algorithm to build a machine that sorts cucumbers by size, shape, color, quality and other features \cite{sato2017tensor}.

%-----------------------------------------------------------------------------------------------------------------------------------------------------------------------------------------------------
\section{Research Objectives} 
An examination of the two case studies above (i.e., robotany used by IntelinAir and Makoto Koike) can help determine weather robotany has and will be distinctly different from past processes of McDonaldization and mechanization. This chapter will discuss the effects of the open-source movement and the availability of rental cloud computing services, that allow for small farmers to use this technology. Finally, it will discuss the irrationalities that robotany may produce or quell for our society.  
%answer the following research questions:
%\begin{enumerate}
%	\item How is the McDonaldization of arable agriculture weathering with the advent of robotany in California? 
%	\item How will the open-source movement and the availability of rental cloud computing services, that allow for small farmers to use this technology, affect this process? 
%	\item What irrationalities may robotany produce that may have been non-existent in past mechanization? And what irrationalities may it take away?
%\end{enumerate}
	
%-----------------------------------------------------------------------------------------------------------------------------------------------------------------------------------------------------
\section{Tentative Hypothesis} 
The difference between past mechanization and robotany is perhaps the extent to which we are using robotany to control nature and therefore the extent to which it can transform farming practices. For the farmer, the specialization of labor (e.g., no multi-cropping, or crop rotations) leading into the mechanization of labor spelled the loss of control over workday, when to plant, when to weed, how to cultivate, etc. Robotany isn't necessarily a new and distinctly different process, but perhaps is exacerbating the loss of control through non-human means.

In many instances, because the machines will learn to optimize farming of one crop at a time (called ``weak AI"), it will likely reproduce certain types of agricultural systems that are less environmentally sustainable and characterized by growing skill gaps between on-the-ground laborers, who are still going to be needed at least a little, and those who are the tech gurus behind the systems. This process is racialized; people of color will more likely be in laborers the field than behind the system. However, ``strong AI" or general AI aims to replace human laborers as a whole; the goal is to mimic the entirety of human intelligence. If AI successfully gets here, it will spell non-specialization of machine labor. Here, humans are entirely superfluous to the farming process and thus the questions concerning on-the-ground human farm labor disappear. 

%-----------------------------------------------------------------------------------------------------------------------------------------------------------------------------------------------------
\section{Conclusion} 
According to Ritzer (2002), the irrationality of rationality is the fifth dimension of McDonaldization. Robotany may produce irrationalities in production (e.g., genetic modification and reduced genetic diversity of seeds, data privacy and security), harvesting (e.g., loss of jobs, waste, and environmental degradation), and consumption (e.g., consumer appetite for uniformity). These irrationalities have important implications for public policy. As it stands, we have two choices: the possibility of a better life with less labor and more leisure time to be creative, or to face mass unemployment and continued wealth and income inequality.




